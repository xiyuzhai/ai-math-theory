%:
\documentclass[11pt, oneside]{article}   	% use "amsart" instead of "article" for AMSLaTeX format
\usepackage{geometry}                		% See geometry.pdf to learn the layout options. There are lots.
\geometry{letterpaper}                   		% ... or a4paper or a5paper or ... 
%\geometry{landscape}                		% Activate for rotated page geometry
%\usepackage[parfill]{parskip}    		% Activate to begin paragraphs with an empty line rather than an indent
\usepackage{graphicx}				% Use pdf, png, jpg, or eps§ with pdflatex; use eps in DVI mode
								% TeX will automatically convert eps --> pdf in pdflatex		
\usepackage{amssymb}
\usepackage{diagbox}
\usepackage{amsmath}
\usepackage{amsthm}
\usepackage{enumerate}
\theoremstyle{definition}
\newtheorem*{defn}{Definition}
\newtheorem*{prop}{Proposition}
\newtheorem*{eg}{Example}
\newtheorem*{thm}{Theorem}
\newtheorem*{corol}{Corollary}
\newtheorem{ex}{Exercise}[section]
{\theoremstyle{plain}
\newtheorem*{rmk}{Remark}
\newtheorem*{rmks}{Remarks}
\newtheorem*{lt}{Last time}
}
\newtheorem*{lem}{Lemma}
\usepackage{color}
\title{AI Math Theory}
\author{Xiyu Zhai}
\date{}							% Activate to display a given date or no date

\begin{document}
\maketitle
\tableofcontents

\section{Introduction}

Recently, the success of transformers have amazed the world by not only trivialize traditional problems in natural language processing, but also exhibits unexpected capabilities in mathematical and common sense reasoning, marking the sparks of general intelligence. People are beginning to expect Artificial General Intelligence to be achieved in the near future. As Mathematics is commonly perceived as the pinnacle of human intelligence, it naturally becomes one of the most important remaining territories for AI to conquer. Very recently, the work AlphaGeometry solves IMO geometry problems as good as a golden medalist, by combining connectism AI and symbolism AI. One can't help imagine what the future could hold.

However, there is a lack of theoretical understanding of the nature of the tasks and what algorithms could be good. The closest suitable theories would be RL theories, and for clarity the RL viewpoint of AI mathematics is
(TODO: make this a table?)
\begin{itemize}
	\item RL state corresponds to proof state, i.e. the collection of facts assumed and derived;
	\item RL reward is whether the ;
	\item policy RL corresponds to tactics
\end{itemize}

 still there are several key differences:
\begin{itemize}
	\item in Mathematics, there is no intrinsic statistical element, i.e. no Markov process. A proof is either correct or wrong, no middeground. In a sense, this is simpler than MDP. Just like Von Neumann once said, "If people do not believe that mathematics is simple, it is only because they do not realize how complicated life is.", we would like to comment that "If people do not believe that AI mathematics is simple, it is only because they do not realize how complicated RL is."
	\item 
\end{itemize}

\section{Related Work}

\section{S}



\end{document}